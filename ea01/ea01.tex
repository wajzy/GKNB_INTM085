\documentclass[usenames,dvipsnames,aspectratio=169]{beamer}
\usepackage{../common/cpp}

\title[OO Programozás - C++]{OO Programozás}
\subtitle{Osztályok}

\begin{document}

%1
\begin{frame}[plain]
  \titlepage
  \logoalul
\end{frame}

\section{Struktúrák}

\begin{frame}
    Struktúrák C-ben:
    \begin{itemize}
        \item[$+$] Logikailag összetartozó \emph{adatok} egységbe zárása
        \item[$-$] Függvények nem lehetnek tagok (de függvényeket címző mutatók igen)
        \item[$-$] Minden adat mindenki számára hozzáférhető
        \item[$-$] A \emph{tagok} inicializálhatók, de el lehet feledkezni róla 
    \end{itemize}
    \vfill
    Hogyan valósítanánk meg téglalapok kerületének és területének kiszámítását a régi eszközökkel?
\end{frame}

\begin{frame}
    \begin{exampleblock}{\textattachfile{Rectangle01.cpp}{Rectangle01.cpp}}
        \lstinputlisting[language=C++,linerange={1-11},numbers=left,firstnumber=1]{Rectangle01.cpp}
    \end{exampleblock}
\end{frame}

\begin{frame}
    \begin{exampleblock}{\textattachfile{Rectangle01.cpp}{Rectangle01.cpp}}
        \scriptsize
        \lstinputlisting[language=C++,linerange={13-26},numbers=left,firstnumber=13]{Rectangle01.cpp}
    \end{exampleblock}
\end{frame}

\begin{frame}[fragile]
    \begin{exampleblock}{\textattachfile{Rectangle01.cpp}{Rectangle01.cpp}}
        \scriptsize
        \lstinputlisting[language=C++,linerange={28-37},numbers=left,firstnumber=28]{Rectangle01.cpp}
    \end{exampleblock}
    \begin{block}{Kimenet}
        \vspace{-.3cm}
        \tiny
        \begin{verbatim}
Rectangle[dimensions: 6.95335e-310 x 0]
Rectangle[dimensions: 5 x 3]
Rectangle[dimensions: 4.5 x 3]
Area: 13.5
Perimeter: 15
\end{verbatim}
    \vspace{-.3cm}
    \end{block}
\end{frame}

\begin{frame}
    Használjuk ki a C++ struktúrák képességeit:
    \begin{itemize}
        \item[$+$] Egységbezárás elve: adatok és műveletek (tagfüggvények) egyetlen struktúrában
        \item[$+$] Névteret képez 
        \item[$+$] Láthatósági kategóriák: \texttt{public} (alapértelmezett), \texttt{private} 
    \end{itemize}
    \vfill
    További változtatások:
    \begin{itemize}
        \item Tagok minősítése (változó\kiemel{.}tagnév)
        \item \emph{const} függvények $\to$ adattagokat nem módosíthatják
    \end{itemize}
\end{frame}

\begin{frame}
    \begin{exampleblock}{\textattachfile{Rectangle02.cpp}{Rectangle02.cpp}}
        \small
        \lstinputlisting[language=C++,linerange={1-12},numbers=left,firstnumber=1]{Rectangle02.cpp}
    \end{exampleblock}
\end{frame}

\begin{frame}
    \begin{exampleblock}{\textattachfile{Rectangle02.cpp}{Rectangle02.cpp}}
        \small
        \lstinputlisting[language=C++,linerange={14-26},numbers=left,firstnumber=14]{Rectangle02.cpp}
    \end{exampleblock}
\end{frame}

\begin{frame}[fragile]
    \begin{exampleblock}{\textattachfile{Rectangle02.cpp}{Rectangle02.cpp}}
        \scriptsize
        \lstinputlisting[language=C++,linerange={35-43},numbers=left,firstnumber=35]{Rectangle02.cpp}
    \end{exampleblock}
    \begin{block}{Kimenet}
        \vspace{-.3cm}
        \tiny
        \begin{verbatim}
Rectangle[dimensions: 5 x 3]
Area: 15
Perimeter: 16
\end{verbatim}
    \vspace{-.3cm}
    \end{block}
\end{frame}

\section{Osztályok}

\begin{frame}
    \emph{Osztályok} (\texttt{class}) C++-ban:
    \begin{itemize}
        \item Új, felhasználói adattípust hoz létre
        \item Adatok és rajtuk végezhető műveletek \emph{egységbe zárása} (encapsulation)
        \item Szigorúbb \emph{adatrejtés}: \texttt{private} (alapértelmezett), \texttt{public} $\to$ hibák elkerülése
        \item Adatszerkezet és a megvalósítás részletei változtathatók, ameddig az interfész (nyilvános függvények) változatlan
        \item \emph{Adattagok} és \emph{tagfüggvények} tekintetében névteret képez
    \end{itemize}
    \vfill
    Osztály: ,,terv'', \kiemel{nincs memóriafoglalás!} \\
    \emph{Objektum} (példány): saját memóriaterület az adatokhoz, de osztoznak a függvényeken (mindig az aktuális példány adatain dolgoznak $\to$ \texttt{this})
\end{frame}

\begin{frame}
    További lehetőségek:
    \begin{itemize}
        \item \texttt{inline} függvények: ha lehet, a fordító a kódot többször is beágyazza annak hívása helyett
        \item Tagfüggvények az osztályon belül (implicit inline) és kívül is definiálhatók
    \end{itemize}    
\end{frame}

\begin{frame}
    \begin{exampleblock}{\textattachfile{Rectangle03.cpp}{Rectangle03.cpp}}
        \footnotesize
        \lstinputlisting[language=C++,linerange={3-17},numbers=left,firstnumber=3]{Rectangle03.cpp}
    \end{exampleblock}
\end{frame}

\begin{frame}
    \begin{exampleblock}{\textattachfile{Rectangle03.cpp}{Rectangle03.cpp}}
        \small
        \lstinputlisting[language=C++,linerange={18-28},numbers=left,firstnumber=18]{Rectangle03.cpp}
    \end{exampleblock}
\end{frame}

\begin{frame}
    \begin{exampleblock}{\textattachfile{Rectangle03.cpp}{Rectangle03.cpp}}
        \small
        \lstinputlisting[language=C++,linerange={30-39},numbers=left,firstnumber=30]{Rectangle03.cpp}
    \end{exampleblock}
\end{frame}

\begin{frame}
    \begin{exampleblock}{\textattachfile{Rectangle03.cpp}{Rectangle03.cpp}}
        \small
        \lstinputlisting[language=C++,linerange={41-47},numbers=left,firstnumber=41]{Rectangle03.cpp}
    \end{exampleblock}
\end{frame}

\begin{frame}
    Az inline függvények alternatívája C-ben: paraméteres makrók (ld. \texttt{cctype}).
    \begin{exampleblock}{\textattachfile{macro-vs-inline.cpp}{macro-vs-inline.cpp}}
        \scriptsize
        \lstinputlisting[language=C++,linerange={3-17},numbers=left,firstnumber=3]{macro-vs-inline.cpp}
    \end{exampleblock}
\end{frame}

\begin{frame}
    \begin{center}
        \begin{tabular}{ p{2.5cm}|p{5cm}|p{5cm} } 
        Szempont & Inline fv. & Makro \\
        \hline
        Felhasználható & \multicolumn{2}{c}{Rövid, gyakran futó kódoknál} \\
        Feldolgozza & fordító & előfeldolgozó \\ 
        Paraméterek kiértékelése & egyszer & minden ,,hívásnál'' \\ 
        Kifejtés & nem garantált & garantált \\ 
        Adattagok elérése & közvetlenül & csak közvetetten (nem tagja osztálynak) \\ 
        Nyomkövetés & egyszerű & nehézkes \\
        Típusok & fordító kikényszeríti a megfelelő típusú paramétereket; biztonságos, de rugalmatlan $\to$ felültöltés, sablonok & nincs típusellenőrzés; rugalmas, de nem biztonságos \\
        \end{tabular}
    \end{center}
\end{frame}

\begin{frame}
    Hatókör feloldó operátor (\texttt{::})
    \begin{itemize}
        \item Adott névtér elemeinek eléréséhez (pl. \texttt{std::cout})
        \item Adott osztály tagfüggvényének külső definiálásakor \\
        (pl. \texttt{void Rectangle::print() \{\dots \}})
        \item Azonos nevű változó által elfedett globális változó eléréséhez
        \item Statikus adattagok eléréséhez (ld. később)
        \item És még néhány további esetben
    \end{itemize}
\end{frame}

\begin{frame}[fragile]
    \begin{columns}
        \begin{column}{0.75\textwidth}
            \begin{exampleblock}{\textattachfile{scope-resolution.cpp}{scope-resolution.cpp}}
                \scriptsize
                \lstinputlisting[language=C++,linerange={1-16},numbers=left,firstnumber=1]{scope-resolution.cpp}
            \end{exampleblock}
        \end{column}
        \begin{column}{0.25\textwidth}
            \begin{block}{Kimenet}
                \begin{verbatim}
global i=1
A/global i=2
local i=3
\end{verbatim}
            \end{block}
        \end{column}
    \end{columns}
\end{frame}

\section{Kódok elrendezése}

\begin{frame}
    Nagyobb programoknál célszerű osztályonként két fájlt létrehozni:
    \begin{itemize}
        \item fejfájlt (\texttt{.h}, \texttt{.hpp}) a deklarációknak, inline függvényeknek, majd védeni többszörös beszerkesztés ellen
        \item forrásfájlt (\texttt{.cpp}, \texttt{.cc}, \texttt{.cxx}) a definícióknak
    \end{itemize}
    Gyorsabb fordítás, kisebb binárisok.
    \begin{exampleblock}{\textattachfile{Rectangle04.h}{Rectangle04.h}}
        \footnotesize
        \lstinputlisting[language=C++,linerange={1-6},numbers=left,firstnumber=1]{Rectangle04.h}
    \end{exampleblock}
\end{frame}

\begin{frame}
    \begin{exampleblock}{\textattachfile{Rectangle04.h}{Rectangle04.h}}
        \footnotesize
        \lstinputlisting[language=C++,linerange={8-21},numbers=left,firstnumber=8]{Rectangle04.h}
    \end{exampleblock}
\end{frame}

\begin{frame}
    \begin{exampleblock}{\textattachfile{Rectangle04.h}{Rectangle04.h}}
        \lstinputlisting[language=C++,linerange={23-29},numbers=left,firstnumber=23]{Rectangle04.h}
    \end{exampleblock}
\end{frame}

\begin{frame}
    \begin{exampleblock}{\textattachfile{Rectangle04.cpp}{Rectangle04.cpp}}
        \small
        \lstinputlisting[language=C++,linerange={1-12},numbers=left,firstnumber=1]{Rectangle04.cpp}
    \end{exampleblock}
\end{frame}

\begin{frame}
    \begin{exampleblock}{\textattachfile{main04.cpp}{main04.cpp}}
        \small
        \lstinputlisting[language=C++,linerange={1-8},numbers=left,firstnumber=1]{main04.cpp}
    \end{exampleblock}
\end{frame}

\section{Konstruktorok}

\begin{frame}
    Régi probléma: inicializálás elfelejthető, megkerülhető $\to$ konstruktor\\
    Neve egyezik az osztályéval, visszatérési érték/típus nincs.
    \begin{exampleblock}{\textattachfile{Rectangle05.h}{Rectangle05.h}}
        \scriptsize
        \lstinputlisting[language=C++,linerange={6-10},numbers=left,firstnumber=6]{Rectangle05.h}
    \end{exampleblock}
    \begin{exampleblock}{\textattachfile{Rectangle05.cpp}{Rectangle05.cpp}}
        \scriptsize
        \lstinputlisting[language=C++,linerange={3-6},numbers=left,firstnumber=3]{Rectangle05.cpp}
    \end{exampleblock}
\end{frame}

\begin{frame}
    \begin{exampleblock}{\textattachfile{main05.cpp}{main05.cpp}}
        \footnotesize
        \lstinputlisting[language=C++,linerange={2-9},numbers=left,firstnumber=2]{main05.cpp}
    \end{exampleblock}
\end{frame}

\begin{frame}
    Alapértelmezett (default) konstruktor:
    \begin{itemize}
        \item nem kapnak paramétereket, vagy mindegyiknek van alapértelmezett értéke
        \item adattagok alapértelmezett értékekkel feltöltésére, többnyire nullázásra
        \item néha létre \emph{kell} hozni (pl. sok objektum dinamikus létrehozásához)
        \item ha nem készítünk konstruktort egy osztályhoz, akkor a fordító létrehoz egy alapértelmezettet (többnyire nem hasznos)
        \item hívásakor ne használjunk zárójelpárt ($\to$ függvény, ami az osztály példányával tér vissza)
    \end{itemize}
    A konstruktorok felüldefiniálhatók.
\end{frame}

\begin{frame}
    Két konstruktor egy osztályban.
    \begin{exampleblock}{\textattachfile{Rectangle06.h}{Rectangle06.h}}
        \lstinputlisting[language=C++,linerange={6-11},numbers=left,firstnumber=6]{Rectangle06.h}
    \end{exampleblock}
\end{frame}

\begin{frame}
    \begin{exampleblock}{\textattachfile{Rectangle06.cpp}{Rectangle06.cpp}}
        \lstinputlisting[language=C++,linerange={3-10},numbers=left,firstnumber=3]{Rectangle06.cpp}
    \end{exampleblock}
\end{frame}

\begin{frame}
    \begin{exampleblock}{\textattachfile{main06.cpp}{main06.cpp}}
        \footnotesize
        \lstinputlisting[language=C++,linerange={2-11},numbers=left,firstnumber=2]{main06.cpp}
    \end{exampleblock}
    \vfill
    \begin{enumerate}
        \item A konstruktor \emph{még} nem ellenőrzi a megadott adatok helyességét.
        \item A két konstruktor összevonható.
    \end{enumerate}
\end{frame}

\begin{frame}
    \begin{exampleblock}{\textattachfile{Rectangle07.h}{Rectangle07.h}}
        \scriptsize
        \lstinputlisting[language=C++,linerange={6-11},numbers=left,firstnumber=6]{Rectangle07.h}
    \end{exampleblock}
    \begin{exampleblock}{\textattachfile{Rectangle07.cpp}{Rectangle07.cpp}}
        \scriptsize
        \lstinputlisting[language=C++,linerange={4-8},numbers=left,firstnumber=4]{Rectangle07.cpp}
    \end{exampleblock}
\end{frame}

\begin{frame}[fragile]
    \begin{columns}
        \begin{column}{0.7\textwidth}
            \begin{exampleblock}{\textattachfile{main07.cpp}{main07.cpp}}
                \scriptsize
                \lstinputlisting[language=C++,linerange={2-12},numbers=left,firstnumber=2]{main07.cpp}
            \end{exampleblock}
        \end{column}
        \begin{column}{0.3\textwidth}
            \begin{block}{Kimenet}
                \vspace{-0.4cm}
                \scriptsize
                \begin{verbatim}
Rectangle[dimensions: 5 x 3]
Area: 15
Perimeter: 16
Rectangle[dimensions: 0 x 10]
Rectangle[dimensions: 5 x 0]
Rectangle[dimensions: 0 x 0]
\end{verbatim}
                \vspace{-0.4cm}
            \end{block}
        \end{column}
    \end{columns}
\end{frame}

\begin{frame}
    Bővítések:
    \begin{itemize}
        \item adatok lekérdezése és ellenőrzött beállítása (getter, setter)
        \item azonos nevű paraméter elrejti az adattagot $\to$ \texttt{this}
    \end{itemize}
    \begin{exampleblock}{\textattachfile{Rectangle08.h}{Rectangle08.h}}
        \scriptsize
        \vspace{-0.3cm}
        \lstinputlisting[language=C++,linerange={7-10},numbers=left,firstnumber=7]{Rectangle08.h}
        \lstinputlisting[language=C++,linerange={13-20},numbers=left,firstnumber=13]{Rectangle08.h}
        \vspace{-0.3cm}
    \end{exampleblock}
\end{frame}

\begin{frame}
    \begin{exampleblock}{\textattachfile{Rectangle08.h}{Rectangle08.h}}
        \lstinputlisting[language=C++,linerange={22-30},numbers=left,firstnumber=22]{Rectangle08.h}
    \end{exampleblock}
\end{frame}

\begin{frame}
    \begin{exampleblock}{\textattachfile{main08.cpp}{main08.cpp}}
        \lstinputlisting[language=C++,linerange={1-8},numbers=left,firstnumber=1]{main08.cpp}
    \end{exampleblock}
\end{frame}

\begin{frame}
    Destruktorok:
    \begin{itemize}
        \item objektum megsemmisülésekor kerül meghívásra
        \item tipikusan a lefoglalt erőforrások felszabadítására használják
        \item neve \kiemel{~} jellel kezdődik, az osztálynevével folytatódik
        \item nincs visszatérési értéke/típusa
        \item nem fogadhat paramétereket
    \end{itemize}
\end{frame}

\begin{frame}
    Mikor hívják a konstruktorokat és destruktorokat?
    \begin{center}
        \begin{tabular}{ p{3cm}|p{4.5cm}|p{4.5cm} } 
        Jelleg & Konstruktor & Destruktor \\
        \hline
        globális obj. & \texttt{main} előtt & \texttt{main} után \\
        lokális obj. & vezérlés az obj. definíciójára kerül & vezérlés elhagyja a definiáló blokkot \\
        lokális statikus obj. & vezérlés az obj. definíciójára kerül & program végén \\
        dinamikus obj. & \texttt{new} hatására & \texttt{delete} hatására
        \end{tabular}
    \end{center}
\end{frame}

\begin{frame}
    Statikus tagok
    \begin{itemize}
        \item minden objektum osztozik ugyanazon az adaton
        \item \texttt{static} kulcsszóval hozható létre
        \item változó definiálása, inicializálása  az osztályon kívül történik
        \item az osztály nevével és hatókör feloldó operátorral érhető el
        \item vagy egy statikus tagfüggvénnyel, ami csak statikus adattagokat érhet el és statikus tagfüggvényeket hívhat
    \end{itemize}
\end{frame}

\begin{frame}
    \begin{exampleblock}{\textattachfile{Rectangle09.h}{Rectangle09.h}}
        \small
        \lstinputlisting[language=C++,linerange={7-13},numbers=left,firstnumber=7]{Rectangle09.h}
        \lstinputlisting[language=C++,linerange={40-43},numbers=left,firstnumber=40]{Rectangle09.h}
    \end{exampleblock}
\end{frame}

\begin{frame}
    \begin{exampleblock}{\textattachfile{Rectangle09.cpp}{Rectangle09.cpp}}
        \vspace{-.3cm}
        \scriptsize
        \lstinputlisting[language=C++,linerange={4-14},numbers=left,firstnumber=4]{Rectangle09.cpp}
        \lstinputlisting[language=C++,linerange={20-25},numbers=left,firstnumber=20]{Rectangle09.cpp}
        \vspace{-.3cm}
    \end{exampleblock}
\end{frame}

\begin{frame}[fragile]
    \begin{exampleblock}{\textattachfile{main09.cpp}{main09.cpp}}
        \scriptsize
        \lstinputlisting[language=C++,linerange={1-5},numbers=left,firstnumber=1]{main09.cpp}
    \end{exampleblock}
    \begin{block}{Kimenet}
        \scriptsize
        \vspace{-.3cm}
        \begin{verbatim}
Reactangle #1 created.
Reactangle #2 created.
Reactangle #3 created.
Number of rectangles: 3
Reactangle #3 freed.
Reactangle #2 freed.
Reactangle #1 freed.
\end{verbatim}
\vspace{-.3cm}
    \end{block}
\end{frame}

\begin{frame}
    Objektumtömbök létrehozása, memóriafoglalás/felszabadítás
    \begin{exampleblock}{\textattachfile{message.cpp}{message.cpp}}
        \scriptsize
        \lstinputlisting[language=C++,linerange={4-17},numbers=left,firstnumber=4]{message.cpp}
    \end{exampleblock}
\end{frame}

\begin{frame}
    \begin{exampleblock}{\textattachfile{message.cpp}{message.cpp}}
        \scriptsize
        \lstinputlisting[language=C++,linerange={19-33},numbers=left,firstnumber=19]{message.cpp}
    \end{exampleblock}
\end{frame}

\begin{frame}
    \begin{exampleblock}{\textattachfile{message.cpp}{message.cpp}}
        \footnotesize
        \lstinputlisting[language=C++,linerange={35-48},numbers=left,firstnumber=35]{message.cpp}
    \end{exampleblock}
\end{frame}

\begin{frame}[fragile]
    \begin{block}{Kimenet}
        \scriptsize
        \vspace{-0.3cm}
        \begin{verbatim}
Created [0x7ffe101e7718]
Hello C++ world!
Created [0xbfa2e0, Object on heap.]
Freed [0xbfa2e0, Object on heap.]
Created [0x7ffe101e7720, alpha]
Created [0x7ffe101e7728, beta]
Created [0xbfa328]
Created [0xbfa330]
Created [0xbfa338]
Freed [0xbfa338, ]
Freed [0xbfa330, ]
Freed [0xbfa328, ]
Freed [0x7ffe101e7728, beta]
Freed [0x7ffe101e7720, alpha]
Freed [0x7ffe101e7718, Hello C++ world!
]
\end{verbatim}
        \vspace{-0.3cm}
    \end{block}
\end{frame}

\end{document}