\section{Néhány hasznos beépített osztály}

\subsection{A \texttt{string} osztály}

\begin{frame}
    Az \texttt{std::string} osztály néhány érdekes tulajdonsága:
    \begin{itemize}
        \item Fejfájl: \texttt{<string>}
        \item Az \hiv{\href{https://en.cppreference.com/w/cpp/string/basic_string}{\texttt{std::basic\_string<char>}}} sablonpéldány szinonimája (\texttt{typedef})
        \item Inicializálható C-stringgel (\texttt{char*}), konverzió visszafelé: \texttt{string::c\_str()}
        \item Dinamikus memóriakezelést használ
        \item Tartalma megváltoztatható (mutable)
        \item Számos felültöltött operátor, pl. összefűzés \texttt{+}, indexelés \texttt{[]}
        \item Megvalósítja a \texttt{RandomAccessIterator}t
    \end{itemize}
\end{frame}

\begin{frame}
    \begin{exampleblock}{\textattachfile{stringDemo.cpp}{stringDemo.cpp}}
        \scriptsize
        \lstinputlisting[language=C++,style=C++,linerange={1-3},numbers=left,firstnumber=1]{stringDemo.cpp}
        \lstinputlisting[language=C++,style=C++,linerange={15-25},numbers=left,firstnumber=15]{stringDemo.cpp}
    \end{exampleblock}
\end{frame}

\begin{frame}[fragile]
    \hiv{\href{https://en.cppreference.com/w/cpp/io/manip/ws}{\texttt{std::ws}}} $\to$ figyelmen kívül hagyja a kezdeti fehér karaktereket
    \begin{block}{Kimenet \#1}
        \vspace{-.4cm}
        \begin{verbatim}
Enter an integer: 5
Enter a line of text: hello
You've entered: 5, hello            
\end{verbatim}
        \vspace{-.3cm}
    \end{block}
    \begin{block}{Kimenet \#2}
        \vspace{-.4cm}
        \begin{verbatim}
Enter an integer: 5cats
Enter a line of text: You've entered: 5, cats
\end{verbatim}
\vspace{-.3cm}
    \end{block}
\end{frame}

\begin{frame}
    \begin{exampleblock}{\textattachfile{stringDemo.cpp}{stringDemo.cpp}}
        \footnotesize
        \lstinputlisting[language=C++,style=C++,linerange={27-39},numbers=left,firstnumber=27]{stringDemo.cpp}
    \end{exampleblock}
\end{frame}

\begin{frame}[fragile]
    \begin{block}{Kimenet \#1}
        \vspace{-.4cm}
        \begin{verbatim}
Enter an integer: 1
Enter a line of text: hello
You've entered: 1, hello
Capacity of the string is 15 chars.
Length of the string is: 5 chars.
Resized and filled with !: hello!!!
Undo resize: hello
Shrinking string... capacity is now 15 chars.        
\end{verbatim}
        \vspace{-.3cm}
    \end{block}
\end{frame}

\begin{frame}[fragile]
    \begin{block}{Kimenet \#2}
        \vspace{-.4cm}
        \begin{verbatim}
Enter an integer: 2
Enter a line of text: TheQuickBrownFoxJumpsOverTheLazyDog
You've entered: 2, TheQuickBrownFoxJumpsOverTheLazyDog
Capacity of the string is 60 chars.
Length of the string is: 35 chars.
Resized and filled with !: TheQuickBrownFoxJumpsOverTheLazyDog!!!
Undo resize: TheQuickBrownFoxJumpsOverTheLazyDog
Shrinking string... capacity is now 35 chars.
\end{verbatim}
        \vspace{-.3cm}
    \end{block}
\end{frame}

\begin{frame}
    \begin{description}[m]
        \item[\hiv{\href{https://en.cppreference.com/w/cpp/string/basic_string/capacity}{\texttt{capacity()}}}] \hfill \\ A jelenleg lefoglalt memóriaterület mérete.
        \item[\hiv{\href{https://en.cppreference.com/w/cpp/string/basic_string/size}{\texttt{length()}}}] \hfill \\ A szöveg hossza.
        \item[\hiv{\href{https://en.cppreference.com/w/cpp/string/basic_string/resize}{\texttt{resize()}}}] \hfill \\ Adott hosszúságúra alakítja a stringet. Ha eredetileg rövidebb volt, megadható, hogy milyen jelekkel töltse fel. Ha hosszabb volt, a végét levágja.
        \item[\hiv{\href{https://en.cppreference.com/w/cpp/string/basic_string/shrink_to_fit}{\texttt{shrink\_to\_fit()}}}] \hfill \\ A lefoglalt, de nem használt memóriaterület felszabadítása iránti igény jelzése. Nem feltétlenül lesz teljesítve.
    \end{description}
\end{frame}

\begin{frame}[fragile]
    \begin{exampleblock}{\textattachfile{stringDemo.cpp}{stringDemo.cpp}}
        \footnotesize
        \lstinputlisting[language=C++,style=C++,linerange={41-45},numbers=left,firstnumber=41]{stringDemo.cpp}
    \end{exampleblock}
    \small
    Az \texttt{std::string\_literals} névtér és az \texttt{s} végződés használatával egy \texttt{std::string} literál formálisan közvetlenül is létrehozható.
    \begin{block}{Kimenet}
        \vspace{-.4cm}
        \footnotesize
        \begin{verbatim}
Enter an integer: 5
Enter a line of text: hello
...
std::string literal length: 11
\end{verbatim}
        \vspace{-.3cm}
    \end{block}
\end{frame}

\begin{frame}
    \begin{exampleblock}{\textattachfile{stringDemo.cpp}{stringDemo.cpp}}
        \scriptsize
        \lstinputlisting[language=C++,style=C++,linerange={47-58},numbers=left,firstnumber=47]{stringDemo.cpp}
    \end{exampleblock}
\end{frame}

\begin{frame}
    \begin{exampleblock}{\textattachfile{stringDemo.cpp}{stringDemo.cpp}}
        \lstinputlisting[language=C++,style=C++,linerange={5-13},numbers=left,firstnumber=5]{stringDemo.cpp}
    \end{exampleblock}
\end{frame}

\begin{frame}[fragile]
    \begin{block}{Kimenet}
        \vspace{-.4cm}
        \begin{verbatim}
Enter an integer: 5
Enter a line of text: hello
...
Displaying the text char-by-char with constant iterator:
hello
Reverse direction:
olleh
Ciphertext: lipps
Plain text: hello
\end{verbatim}
        \vspace{-.3cm}
    \end{block}
\end{frame}

\begin{frame}
    Iterátorok:
    \begin{description}[m]
        \item[\hiv{\href{https://en.cppreference.com/w/cpp/string/basic_string/begin}{\texttt{begin()}, \texttt{cbegin()}}}] \hfill \\ A string elejére mutató (konstans) iterátor.
        \item[\hiv{\href{https://en.cppreference.com/w/cpp/string/basic_string/end}{\texttt{end()}, \texttt{cend()}}}] \hfill \\ A string végére mutató (konstans) iterátor.
        \item[\hiv{\href{https://en.cppreference.com/w/cpp/string/basic_string/rbegin}{\texttt{rbegin()}, \texttt{crbegin()}}}] \hfill \\ A string végére mutató (konstans) iterátor, bejárás fordított irányban.
        \item[\hiv{\href{https://en.cppreference.com/w/cpp/string/basic_string/rend}{\texttt{rend()}, \texttt{crend()}}}] \hfill \\ A string elejére mutató (konstans) iterátor, bejárás fordított irányban.
    \end{description}
    \vfill
    \hiv{\href{https://hu.wikipedia.org/wiki/Caesar-rejtjel}{Caesar-rejtjelezés}}
\end{frame}

\begin{frame}
    \begin{exampleblock}{\textattachfile{stringDemo.cpp}{stringDemo.cpp}}
        \small
        \vspace{-.2cm}
        \lstinputlisting[language=C++,style=C++,linerange={60-72},numbers=left,firstnumber=60]{stringDemo.cpp}
        \vspace{-.2cm}
    \end{exampleblock}
\end{frame}

\begin{frame}[fragile]
    \begin{block}{Kimenet}
        \vspace{-.4cm}
        \begin{verbatim}
Enter an integer: 42
Enter a line of text: Reverse engineering
...
Indices of letter 'e':
1 3 6 8 13 14 
First half: Reverse e
Inserting: <Reverse engineering>
Erasing: Reverse engineering
\end{verbatim}
        \vspace{-.3cm}
    \end{block}
\end{frame}

\begin{frame}
    \begin{description}[m]
        \item[\hiv{\href{https://en.cppreference.com/w/cpp/string/basic_string/find}{\texttt{find(str, pos)}}}, \hiv{\href{https://en.cppreference.com/w/cpp/string/basic_string/rfind}{\texttt{rfind(str, pos)}}}] \hfill \\ Rész-karakterlánc (\texttt{str}) keresése adott helyről (\texttt{pos}) indulva. Ha nincs találat, \hiv{\href{https://en.cppreference.com/w/cpp/string/basic_string/npos}{\texttt{npos}}}-sal tér vissza.
        \item[\hiv{\href{https://en.cppreference.com/w/cpp/string/basic_string/substr}{\texttt{substr(pos, count)}}}] \hfill \\ Rész-karakterlánc előállítása \texttt{pos} helyről indulva, \texttt{count} hosszban.
        \item[\hiv{\href{https://en.cppreference.com/w/cpp/string/basic_string/insert}{\texttt{insert(index, s)}}}] \hfill \\ Az \texttt{s} karakterlánc beszúrása \texttt{index} helyre.
        \item[\hiv{\href{https://en.cppreference.com/w/cpp/string/basic_string/erase}{\texttt{erase(index, count)}}}] \hfill \\ Karakterek törlése az \texttt{index} helytől kezdve, \texttt{count} mennyiségben.
    \end{description}
\end{frame}