\section{Néhány hasznos beépített osztály}

\subsection{A \texttt{string} osztály}

\begin{frame}
    Az \texttt{std::string} osztály néhány érdekes tulajdonsága:
    \begin{itemize}
        \item Fejfájl: \texttt{<string>}
        \item Az \hiv{\href{https://en.cppreference.com/w/cpp/string/basic_string}{\texttt{std::basic\_string<char>}}} sablonpéldány szinonimája (\texttt{typedef})
        \item Inicializálható C-stringgel (\texttt{char*}), konverzió visszafelé: \texttt{string::c\_str()}
        \item Dinamikus memóriakezelést használ
        \item Tartalma megváltoztatható (mutable)
        \item Számos felültöltött operátor, pl. összefűzés \texttt{+}, indexelés \texttt{[]}
        \item Megvalósítja a \texttt{RandomAccessIterator}t
    \end{itemize}
\end{frame}

\begin{frame}
    \begin{exampleblock}{\textattachfile{stringDemo.cpp}{stringDemo.cpp}}
        \scriptsize
        \lstinputlisting[language=C++,style=C++,linerange={1-3},numbers=left,firstnumber=1]{stringDemo.cpp}
        \lstinputlisting[language=C++,style=C++,linerange={15-25},numbers=left,firstnumber=15]{stringDemo.cpp}
    \end{exampleblock}
\end{frame}

\begin{frame}[fragile]
    \hiv{\href{https://en.cppreference.com/w/cpp/io/manip/ws}{\texttt{std::ws}}} $\to$ figyelmen kívül hagyja a kezdeti fehér karaktereket
    \begin{block}{Kimenet \#1}
        \vspace{-.4cm}
        \begin{verbatim}
Enter an integer: 5
Enter a line of text: hello
You've entered: 5, hello            
\end{verbatim}
        \vspace{-.3cm}
    \end{block}
    \begin{block}{Kimenet \#2}
        \vspace{-.4cm}
        \begin{verbatim}
Enter an integer: 5cats
Enter a line of text: You've entered: 5, cats
\end{verbatim}
\vspace{-.3cm}
    \end{block}
\end{frame}

\begin{frame}
    \begin{exampleblock}{\textattachfile{stringDemo.cpp}{stringDemo.cpp}}
        \footnotesize
        \lstinputlisting[language=C++,style=C++,linerange={27-39},numbers=left,firstnumber=27]{stringDemo.cpp}
    \end{exampleblock}
\end{frame}

\begin{frame}[fragile]
    \begin{block}{Kimenet \#1}
        \vspace{-.4cm}
        \begin{verbatim}
Enter an integer: 1
Enter a line of text: hello
You've entered: 1, hello
Capacity of the string is 15 chars.
Length of the string is: 5 chars.
Resized and filled with !: hello!!!
Undo resize: hello
Shrinking string... capacity is now 15 chars.        
\end{verbatim}
        \vspace{-.3cm}
    \end{block}
\end{frame}

\begin{frame}[fragile]
    \begin{block}{Kimenet \#2}
        \vspace{-.4cm}
        \begin{verbatim}
Enter an integer: 2
Enter a line of text: TheQuickBrownFoxJumpsOverTheLazyDog
You've entered: 2, TheQuickBrownFoxJumpsOverTheLazyDog
Capacity of the string is 60 chars.
Length of the string is: 35 chars.
Resized and filled with !: TheQuickBrownFoxJumpsOverTheLazyDog!!!
Undo resize: TheQuickBrownFoxJumpsOverTheLazyDog
Shrinking string... capacity is now 35 chars.
\end{verbatim}
        \vspace{-.3cm}
    \end{block}
\end{frame}

\begin{frame}
    \begin{description}[m]
        \item[\hiv{\href{https://en.cppreference.com/w/cpp/string/basic_string/capacity}{\texttt{capacity()}}}] \hfill \\ A jelenleg lefoglalt memóriaterület mérete.
        \item[\hiv{\href{https://en.cppreference.com/w/cpp/string/basic_string/size}{\texttt{length()}}}] \hfill \\ A szöveg hossza.
        \item[\hiv{\href{https://en.cppreference.com/w/cpp/string/basic_string/resize}{\texttt{resize()}}}] \hfill \\ Adott hosszúságúra alakítja a stringet. Ha eredetileg rövidebb volt, megadható, hogy milyen jelekkel töltse fel. Ha hosszabb volt, a végét levágja.
        \item[\hiv{\href{https://en.cppreference.com/w/cpp/string/basic_string/shrink_to_fit}{\texttt{shrink\_to\_fit()}}}] \hfill \\ A lefoglalt, de nem használt memóriaterület felszabadítása iránti igény jelzése. Nem feltétlenül lesz teljesítve.
    \end{description}
\end{frame}

\begin{frame}[fragile]
    \begin{exampleblock}{\textattachfile{stringDemo.cpp}{stringDemo.cpp}}
        \footnotesize
        \lstinputlisting[language=C++,style=C++,linerange={41-45},numbers=left,firstnumber=41]{stringDemo.cpp}
    \end{exampleblock}
    \small
    Az \texttt{std::string\_literals} névtér és az \texttt{s} végződés használatával egy \texttt{std::string} literál formálisan közvetlenül is létrehozható.
    \begin{block}{Kimenet}
        \vspace{-.4cm}
        \footnotesize
        \begin{verbatim}
Enter an integer: 5
Enter a line of text: hello
...
std::string literal length: 11
\end{verbatim}
        \vspace{-.3cm}
    \end{block}
\end{frame}

\begin{frame}
    \begin{exampleblock}{\textattachfile{stringDemo.cpp}{stringDemo.cpp}}
        \scriptsize
        \lstinputlisting[language=C++,style=C++,linerange={47-58},numbers=left,firstnumber=47]{stringDemo.cpp}
    \end{exampleblock}
\end{frame}

\begin{frame}
    \begin{exampleblock}{\textattachfile{stringDemo.cpp}{stringDemo.cpp}}
        \lstinputlisting[language=C++,style=C++,linerange={5-13},numbers=left,firstnumber=5]{stringDemo.cpp}
    \end{exampleblock}
\end{frame}

\begin{frame}[fragile]
    \begin{block}{Kimenet}
        \vspace{-.4cm}
        \begin{verbatim}
Enter an integer: 5
Enter a line of text: hello
...
Displaying the text char-by-char with constant iterator:
hello
Reverse direction:
olleh
Ciphertext: lipps
Plain text: hello
\end{verbatim}
        \vspace{-.3cm}
    \end{block}
\end{frame}

\begin{frame}
    Iterátorok:
    \begin{description}[m]
        \item[\hiv{\href{https://en.cppreference.com/w/cpp/string/basic_string/begin}{\texttt{begin()}, \texttt{cbegin()}}}] \hfill \\ A string elejére mutató (konstans) iterátor.
        \item[\hiv{\href{https://en.cppreference.com/w/cpp/string/basic_string/end}{\texttt{end()}, \texttt{cend()}}}] \hfill \\ A string végére mutató (konstans) iterátor.
        \item[\hiv{\href{https://en.cppreference.com/w/cpp/string/basic_string/rbegin}{\texttt{rbegin()}, \texttt{crbegin()}}}] \hfill \\ A string végére mutató (konstans) iterátor, bejárás fordított irányban.
        \item[\hiv{\href{https://en.cppreference.com/w/cpp/string/basic_string/rend}{\texttt{rend()}, \texttt{crend()}}}] \hfill \\ A string elejére mutató (konstans) iterátor, bejárás fordított irányban.
    \end{description}
    \vfill
    \hiv{\href{https://hu.wikipedia.org/wiki/Caesar-rejtjel}{Caesar-rejtjelezés}}
\end{frame}

\begin{frame}
    \begin{exampleblock}{\textattachfile{stringDemo.cpp}{stringDemo.cpp}}
        \small
        \vspace{-.2cm}
        \lstinputlisting[language=C++,style=C++,linerange={60-72},numbers=left,firstnumber=60]{stringDemo.cpp}
        \vspace{-.2cm}
    \end{exampleblock}
\end{frame}

\begin{frame}[fragile]
    \begin{block}{Kimenet}
        \vspace{-.4cm}
        \begin{verbatim}
Enter an integer: 42
Enter a line of text: Reverse engineering
...
Indices of letter 'e':
1 3 6 8 13 14 
First half: Reverse e
Inserting: <Reverse engineering>
Erasing: Reverse engineering
\end{verbatim}
        \vspace{-.3cm}
    \end{block}
\end{frame}

\begin{frame}
    \begin{description}[m]
        \item[\hiv{\href{https://en.cppreference.com/w/cpp/string/basic_string/find}{\texttt{find(str, pos)}}}, \hiv{\href{https://en.cppreference.com/w/cpp/string/basic_string/rfind}{\texttt{rfind(str, pos)}}}] \hfill \\ Rész-karakterlánc (\texttt{str}) keresése adott helyről (\texttt{pos}) indulva. Ha nincs találat, \hiv{\href{https://en.cppreference.com/w/cpp/string/basic_string/npos}{\texttt{npos}}}-sal tér vissza.
        \item[\hiv{\href{https://en.cppreference.com/w/cpp/string/basic_string/substr}{\texttt{substr(pos, count)}}}] \hfill \\ Rész-karakterlánc előállítása \texttt{pos} helyről indulva, \texttt{count} hosszban.
        \item[\hiv{\href{https://en.cppreference.com/w/cpp/string/basic_string/insert}{\texttt{insert(index, s)}}}] \hfill \\ Az \texttt{s} karakterlánc beszúrása \texttt{index} helyre.
        \item[\hiv{\href{https://en.cppreference.com/w/cpp/string/basic_string/erase}{\texttt{erase(index, count)}}}] \hfill \\ Karakterek törlése az \texttt{index} helytől kezdve, \texttt{count} mennyiségben.
    \end{description}
\end{frame}

\subsection{A \texttt{vector} osztály}

\begin{frame}
    Standard Template Library (STL)
    \small
    \begin{description}[m]
        \item[Gyűjtemények (containers)] \hfill \\ dinamikus tömb (\texttt{vector}), láncolt lista (\texttt{list}, \texttt{forward\_list}), leképezés (asszociatív tömb, szótár: \texttt{map}), halmaz (\texttt{set}), vermek és sorok (\texttt{stack}, \texttt{deque}), stb.
        \item[Algoritmusok] \hfill \\ pl. keresés, csere. 
        \item[Iterátorok] \hfill \\ elemek bejárása különféle irányokban és módokon
        \item[\emph{Függvény objektumok} (functors)] \hfill \\ Függvények paramétereként átadható objektumok. Ezek tagfüggvényeivel a hívott függvény viselkedése befolyásolható, pl. egy gyűjtemény elemei átalakíthatók a kívánt (paraméterezésnek megfelelő) módon.
        \item[\emph{Adapterek} (adapters)] \hfill \\ Más komponensek viselkedését módosítják, pl. egy iterátor bejárási irányát megfordítja. 
    \end{description}
\end{frame}

\begin{frame}
    A \texttt{vector} osztály
    \begin{itemize}
        \item Fejfájl: \texttt<vector>
        \item Dinamikus tömb osztálysablon
        \item Speciális eset: \texttt{vector<bool>} $\to$ bithalmaz
        \item Tartalma megváltoztatható (mutable)
        \item Véletlen elérés: $\mathcal{O}(1)$, beszúrás/törlés a végén: amortizált $\mathcal{O}(1)$, tetszőleges helyen $\mathcal{O}(n)$.
    \end{itemize}
\end{frame}

\begin{frame}
    \begin{exampleblock}{\textattachfile{vectorDemo.cpp}{vectorDemo.cpp}}
        \scriptsize
        \lstinputlisting[language=C++,style=C++,linerange={1-2},numbers=left,firstnumber=1]{vectorDemo.cpp}
        \lstinputlisting[language=C++,style=C++,linerange={12-23},numbers=left,firstnumber=12]{vectorDemo.cpp}
    \end{exampleblock}
\end{frame}

\begin{frame}
    \begin{exampleblock}{\textattachfile{vectorDemo.cpp}{vectorDemo.cpp}}
        \small
        \lstinputlisting[language=C++,style=C++,linerange={25-35},numbers=left,firstnumber=25]{vectorDemo.cpp}
    \end{exampleblock}
\end{frame}

\begin{frame}
    \begin{exampleblock}{\textattachfile{vectorDemo.cpp}{vectorDemo.cpp}}
        \footnotesize
        \vspace{-.2cm}
        \lstinputlisting[language=C++,style=C++,linerange={37-51},numbers=left,firstnumber=37]{vectorDemo.cpp}
        \vspace{-.2cm}
    \end{exampleblock}
\end{frame}

\begin{frame}[fragile]
    \begin{block}{Kimenet}
        \vspace{-.4cm}
        \begin{verbatim}
1 2 3 4 5 
6 7 8 9 10 
42 42 42 42 42 
0 0 0 0 0 
11 12 13 
1 1 1 
1 1 1
\end{verbatim}
        \vspace{-.3cm}
    \end{block}
\end{frame}

\begin{frame}
    Inicializálás, példányosítás
    \begin{itemize}
        \item inicializáló listával
        \item explicit \hiv{\href{https://en.cppreference.com/w/cpp/container/vector/vector}{konstruktor}} hívással, pl.
        \begin{itemize}
            \item \texttt{vector();}
            \item \texttt{explicit vector( size\_type count );}
            \item \texttt{vector( size\_type count, const T\& value, const Allocator\& alloc = Allocator() );}
        \end{itemize}
    \end{itemize}
\end{frame}

\begin{frame}
    \begin{exampleblock}{\textattachfile{vectorDemo.cpp}{vectorDemo.cpp}}
        \scriptsize
        \lstinputlisting[language=C++,style=C++,linerange={53-63},numbers=left,firstnumber=53]{vectorDemo.cpp}
    \end{exampleblock}
\end{frame}

\begin{frame}
    \begin{exampleblock}{\textattachfile{vectorDemo.cpp}{vectorDemo.cpp}}
        \lstinputlisting[language=C++,style=C++,linerange={4-10},numbers=left,firstnumber=4]{vectorDemo.cpp}
    \end{exampleblock}
\end{frame}

\begin{frame}[fragile]
    \begin{block}{Kimenet}
        \vspace{-.4cm}
        \begin{verbatim}
Size of iv6: 5
Capacity: 5
Reserving memory... Capacity: 16
Is it empty? No
Shrinking... 1 2 3 
Growing... 1 2 3 0 0 0 
Further growing... 1 2 3 0 0 0 -1 -1 -1 
Shrinking... Capacity: 9            
\end{verbatim}
        \vspace{-.3cm}
    \end{block}
\end{frame}

\begin{frame}
    \begin{description}[m]
        \item[\hiv{\href{https://en.cppreference.com/w/cpp/container/vector/size}{\texttt{size()}}}] \hfill \\ Tárolt elemek száma.
        \item[\hiv{\href{https://en.cppreference.com/w/cpp/container/vector/capacity}{\texttt{capacity()}}}] \hfill \\ Ennyi elem számára van lefoglalva memóriaterület.
        \item[\hiv{\href{https://en.cppreference.com/w/cpp/container/vector/reserve}{\texttt{reserve(new\_cap)}}}] \hfill \\ Legalább \texttt{new\_cap} elemszámú adatnak foglal le memóriaterületet.
        \item[\hiv{\href{https://en.cppreference.com/w/cpp/container/vector/empty}{\texttt{empty()}}}] \hfill \\ Logikai igaz értékkel tér vissza, ha a vektor üres.
        \item[\hiv{\href{https://en.cppreference.com/w/cpp/container/vector/resize}{\texttt{resize(count)}, \texttt{resize(count, value)}}}] \hfill \\ Ha \texttt{count} kisebb \texttt{size()}-nál, akkor az utolsó elemeket levágja. Ha nagyobb, akkor kibővíti a vektort és a \texttt{value} másolataival tölti fel az új elemeket.
        \item[\hiv{\href{https://en.cppreference.com/w/cpp/container/vector/shrink_to_fit}{\texttt{shrink\_to\_fit()}}}] \hfill \\ Kezdeményezi (de nem feltétlenül hajtja végre) a kihasználatlan kapacitások törlését.
    \end{description}
\end{frame}

\begin{frame}
    \begin{exampleblock}{\textattachfile{vectorDemo.cpp}{vectorDemo.cpp}}
        \scriptsize
        \lstinputlisting[language=C++,style=C++,linerange={65-78},numbers=left,firstnumber=65]{vectorDemo.cpp}
    \end{exampleblock}
\end{frame}

\begin{frame}[fragile]
    \begin{block}{Kimenet}
        \vspace{-.4cm}
        \begin{verbatim}
First element: 1
Last element: -1
Element at idx. 1: 2
Element at idx. 2: 3
Elements in reverse order (using pointers): -1 -1 -1 0 0 0 3 2 1 
Elements in reverse order (using rev. it.): -1 -1 -1 0 0 0 3 2 1             
\end{verbatim}
        \vspace{-.3cm}
    \end{block}
\end{frame}

\begin{frame}
    \begin{description}[m]
        \item[\hiv{\href{https://en.cppreference.com/w/cpp/container/vector/front}{\texttt{front()}}}, \hiv{\href{https://en.cppreference.com/w/cpp/container/vector/back}{\texttt{back()}}}] \hfill \\ Visszaadják az első és utolsó tárolt elemet.
        \item[\hiv{\href{https://en.cppreference.com/w/cpp/container/vector/operator_at}{\texttt{operator[](pos)}}}, \hiv{\href{https://en.cppreference.com/w/cpp/container/vector/at}{\texttt{at(pos)}}}] \hfill \\ Visszaadják a \texttt{pos} indexű elemet.
        \item[\hiv{\href{https://en.cppreference.com/w/cpp/container/vector/data}{\texttt{data()}}}] \hfill \\ Visszaadja a lefoglalt memóriaterület kezdőcímét.
        \item[\hiv{\href{https://en.cppreference.com/w/cpp/container/vector/rbegin}{\texttt{crbegin()}}}, \hiv{\href{https://en.cppreference.com/w/cpp/container/vector/rend}{\texttt{crend()}}}] \hfill \\ Csak olvasásra alkalmas iterátorok az elemek fordított sorrendben történő eléréséhez.
    \end{description}
\end{frame}

\begin{frame}
    \begin{exampleblock}{\textattachfile{vectorDemo.cpp}{vectorDemo.cpp}}
        \scriptsize
        \lstinputlisting[language=C++,style=C++,linerange={80-90},numbers=left,firstnumber=80]{vectorDemo.cpp}
    \end{exampleblock}
\end{frame}

\begin{frame}[fragile]
    \begin{block}{Kimenet}
        \vspace{-.4cm}
        \begin{verbatim}
Assigning... 1 1 1 
Pushing... 1 1 1 2 
Popping... 1 1 1 
Inserting... 1 1 1 2 
Erasing... 1 1 1 
Swapping vectors...
    iv7: 1 2 3 0 0 0 -1 -1 -1 
    iv6: 1 1 1         
\end{verbatim}
        \vspace{-.3cm}
    \end{block}
\end{frame}

\begin{frame}
    \begin{description}[m]
        \item[\hiv{\href{https://en.cppreference.com/w/cpp/container/vector/assign}{\texttt{assign(count, value)}}}] \hfill \\ Lecseréli a vektor elemeit \texttt{count} darab \texttt{value}-ra.
        \item[\hiv{\href{https://en.cppreference.com/w/cpp/container/vector/push_back}{\texttt{push\_back(value)}}}, \hiv{\href{https://en.cppreference.com/w/cpp/container/vector/pop_back}{\texttt{pop\_back()}}}] \hfill \\ Hozzáfűznek vagy eltávolítanak egy elemet a vektor végéhez, -ről.
        \item[\hiv{\href{https://en.cppreference.com/w/cpp/container/vector/insert}{\texttt{insert(pos, value)}}}, \hiv{\href{https://en.cppreference.com/w/cpp/container/vector/erase}{\texttt{erase(pos)}}}] \hfill \\ \texttt{value} beszúrása a \texttt{pos} iterátorral adott helyre, vagy egy elem törlése onnan.
        \item[\hiv{\href{https://en.cppreference.com/w/cpp/container/vector/swap}{\texttt{swap(other)}}}] \hfill \\ Két vektor elemeinek és lefoglalt tárterületének felcserélése.
    \end{description}
\end{frame}

\subsection{A \texttt{deque} osztály}

\begin{frame}
    A \texttt{deque} osztály
    \begin{itemize}
        \item Fejfájl: \texttt<deque>
        \item Hasonló a dinamikus tömbhöz, de lehetővé teszi a gyors ($\mathcal{O}(1)$) beszúrást és törlést a gyűjtemény mindkét végén. Ezt jellemzően úgy éri el, hogy az elemeket nem összefüggő memóriaterületen tárolja, hanem több, egymástól függetlenül lefoglalt memóriaterületen.
        \item Véletlen elérés: $\mathcal{O}(1)$, beszúrás vagy törlés a széleket leszámítva $\mathcal{O}(n)$.
        \item Tartalma megváltoztatható (mutable).
    \end{itemize}
\end{frame}

\begin{frame}
    \begin{exampleblock}{\textattachfile{dequeDemo.cpp}{dequeDemo.cpp}}
        \lstinputlisting[language=C++,style=C++,linerange={1-10},numbers=left,firstnumber=1]{dequeDemo.cpp}
    \end{exampleblock}
\end{frame}

\begin{frame}
    \begin{exampleblock}{\textattachfile{dequeDemo.cpp}{dequeDemo.cpp}}
        \small
        \lstinputlisting[language=C++,style=C++,linerange={12-23},numbers=left,firstnumber=12]{dequeDemo.cpp}
    \end{exampleblock}
\end{frame}

\begin{frame}[fragile]
    \begin{block}{Kimenet}
        \vspace{-.4cm}
        \begin{verbatim}
1 2 4 8 
2 4
\end{verbatim}
        \vspace{-.3cm}
    \end{block}
    \vfill
    \begin{description}[m]
        \item[\hiv{\href{https://en.cppreference.com/w/cpp/container/deque/push_front}{\texttt{push\_front()}}}, \hiv{\href{https://en.cppreference.com/w/cpp/container/deque/pop_front}{\texttt{pop\_front()}}}] \hfill \\ Beszúrás / törlés a sor elején.
    \end{description}
\end{frame}

\subsection{A \texttt{list} osztály}

\begin{frame}
    A \texttt{list} osztály
    \begin{itemize}
        \item Fejfájl: \texttt<list>
        \item Két irányban láncolt lista. Egy irányban láncolt változat: \hiv{\href{https://en.cppreference.com/w/cpp/container/forward_list}{\texttt{forward\_list}}}
        \item Gyors ($\mathcal{O}(1)$) beszúrás és törlés a gyűjtemény bármely pontján, de az elemek véletlen elérése nem támogatott, a bejárás lassú.
        \item Tartalma megváltoztatható (mutable).
    \end{itemize}
\end{frame}

\begin{frame}
    \begin{exampleblock}{\textattachfile{listDemo.cpp}{listDemo.cpp}}
        \lstinputlisting[language=C++,style=C++,linerange={1-10},numbers=left,firstnumber=1]{listDemo.cpp}
    \end{exampleblock}
\end{frame}

\begin{frame}
    \begin{exampleblock}{\textattachfile{listDemo.cpp}{listDemo.cpp}}
        \small
        \lstinputlisting[language=C++,style=C++,linerange={12-21},numbers=left,firstnumber=12]{listDemo.cpp}
    \end{exampleblock}
\end{frame}

\begin{frame}[fragile]
    \begin{block}{Kimenet}
        \vspace{-.4cm}
        \begin{verbatim}
Square numbers: 0 1 4 9 16 25 36
\end{verbatim}
        \vspace{-.3cm}
    \end{block}
    \vfill
    \begin{description}[m]
        \item[\hiv{\href{https://en.cppreference.com/w/cpp/iterator/advance}{\texttt{advance(it, n)}}}] \hfill \\ Lépteti az \texttt{it} iterátort \texttt{n} elemmel. (Ha az iterátor mindkét irányban bejárható, akkor \texttt{n} értéke lehet negatív.)
    \end{description}
\end{frame}

\begin{frame}
    \begin{exampleblock}{\textattachfile{listDemo.cpp}{listDemo.cpp}}
        \small
        \lstinputlisting[language=C++,style=C++,linerange={22-32},numbers=left,firstnumber=22]{listDemo.cpp}
    \end{exampleblock}
\end{frame}

\begin{frame}[fragile]
    \begin{block}{Kimenet}
        \small
        \vspace{-.4cm}
        \begin{verbatim}
In reverse order: 36 25 16 9 4 1 0 
After sorting: 0 1 4 9 16 25 36 
After merging: -3 -2 -2 -2 -1 0 1 4 9 16 25 36 
Removed duplicates: -3 -2 -1 0 1 4 9 16 25 36
\end{verbatim}
        \vspace{-.3cm}
    \end{block}
    \vfill
    \begin{description}[m]
        \small
        \item[\hiv{\href{https://en.cppreference.com/w/cpp/container/list/reverse}{\texttt{reverse()}}}] \hfill \\ Megfordítja az elemek sorrendjét.
        \item[\hiv{\href{https://en.cppreference.com/w/cpp/container/list/sort}{\texttt{sort()}}}] \hfill \\ Növekvő sorrendbe rendezi az elemeket.
        \item[\hiv{\href{https://en.cppreference.com/w/cpp/container/list/merge}{\texttt{merge(list)}}}] \hfill \\ Átmozgatja \texttt{list} tartalmát az aktuális listába. Ha mindkettő rendezett, akkor az eredmény is az lesz.
        \item[\hiv{\href{https://en.cppreference.com/w/cpp/container/list/unique}{\texttt{unique()}}}] \hfill \\ Az egymást követő azonos értékek közül csak az elsőt hagyja meg.
    \end{description}
\end{frame}